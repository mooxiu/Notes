\chapter{Logic and Proofs}

\section{First-order logic}

First-order logic -- also called \textbf{predicate logic}, \textbf{predicate calculus}, \textbf{quantificational logic} (\href{https://en.wikipedia.org/wiki/First-order_logic}{wikipedia}).   

\begin{definition}[Proposition]
    A \textbf{proposition} is a declarative sentence (that is, a sentence that declares a fact) that is either true or false, but not both.
\end{definition}

\begin{eg}
    Examples of proposition:

    \begin{itemize}
        \item Washington, D.C., is the capital of the United States of America.
        \item Toronto is the capital of Canada.
        \item \(1 + 1 = 2\).
        \item \(2 + 2 = 3\).  
    \end{itemize}
\end{eg}


Propositional logic cannot adequately express the meaning of all statements in mathematics and in natural language.

\begin{eg}
    Suppose we know:

    "Every computer connected to the university network is functioning properly".

    No rules of propositional logic allow us to conclude the truth of the statement:

    "MATH3 is functioning properly".

    where MATH3 is one of the computers connected to the university network.
\end{eg}