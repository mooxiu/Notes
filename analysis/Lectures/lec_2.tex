\chapter{Linear Equations of Higher Order}

\section{Introduction: Second-Order Linear Equations}
\begin{definition}[Second-Order Linear DE]
    A second-order differential equation in the (unknown) function \(y(x)\) is one of the form:
    \[
        G(x, y, y', y'') = 0
    \]     

    The DE is said to be \textbf{linear} provided that \(G\) is linear in the dependent variable \(y\) and its derivatives \(y'\) and \(y''\).   

    Thus a linear second-order equation takes the form:
    \[
        A(x) y'' + B(x) y' + C(x) y = F(x)
    \]

    We don't require that \(A(x), B(x), C(x)\) and \(F(x)\) be linear functions of \(x\).   
\end{definition}

\begin{remark}[Linear]
     This is linear:
     \[
        e^xy'' + (cos x) y' + (1 + \sqrt{x}) y = tan^{-1} x
     \]

    These are not linear:
    \[
        y'' = y y'
    \]
    \[
        y'' + 3(y')^2 + 4y^3 = 0
    \]
\end{remark}

\begin{example}[Homogeneous and Nonhomogeneous]
   This is nonhomogeneous:
   \[
    x^2 y'' + 2xy' + 3y = cos x
   \] 

    Its \textit{associated} homogeneous equation is:
    \[
        x^2 y'' + 2xy' + 3y = 0
    \]

    In general, the homogeneous linear equation associated with the general form is :
    \[
        A(x) y'' + B(x) y' + C(x) y = 0 
    \]
\end{example}

The importance of homogeneous linear equation is that the sum of any two solutions is again a solution, as is any constant multiple of a solution. 

\begin{theorem} [Superposition for Homogeneous Equations]
    For DE:
    \[
        y'' + p(x) y' + q(x) y = 0
    \]

    If we have \(y_1\) and \(y_2\) being two solutions for the above DE,   
    the linear combination:
    \[
        y = c_1y_1 + c_2y_2
    \]
    is also a solution for the DE.
\end{theorem}

\begin{theorem}[Existence and Uniqueness for Linear Equations]
    Suppose that the functions \(p\), \(q\) and \(f\) are continuous on the open interval \(I\)  containing the point \(a\).  
    Then given any 2 numbers \(b_0\) and \(b_1\), the equation:
    \[
        y'' + p(x)y' + q(x)y = f(x)
    \]
    
    has a unique (that is, one and only one) solution on the entire interval \(I\) that satisfies the initial conditions:
    \[
        y(a) = b_0, \quad y'(a) = b_1
    \] 
\end{theorem}

\begin{example}
    Verify that the functions
    \[
        y_1(x) = e^x \quad and \quad y_2(x) = x e^x
    \]
    are the solutions of the differential equation
    \[
        y'' - 2y' + y = 0
    \]
    and then find a solution satisfying the initial conditions \( y(0) = 3\), \(y'(0) = 1\)  
\end{example}
\begin{proof}[solution]
    Verify \(y_1\):
    \[
        y'' - 2y' + y = e^x - 2e^x + e^x = 0
    \] 

    Verify \(y_2\):
    \[
        y'' - 2y' + y = (x + 2)e^x - 2(x+1)e^x + xe^x = 0
    \] 

    The general solution will be:
    \[
        y = c_1e^x + c_2 xe^x
    \]

    Now we can bring in the initial conditions:
    \begin{align*}
        y(0) &= c_1 = 3  \\
        y'(0) &= c_1 + c_2 = 1
    \end{align*}

    Thus we have \(y = 3e^x -2xe^x\) 
\end{proof}

\begin{definition}[Linear Independence of Two Functions]
    Two functions defined on an open interval \(I\) are said to be \textbf{linearly independent} on \(I\) provided that neither is a constant multiple of the other.   
\end{definition}

Based on the second theorem, the homogeneous equation always has 2 linear independent solutions, therefore we have a general solution:
\[
    y = c_1y_1 + c_2y_2
\]

Bring 2 initial conditions, we can have:
\begin{align*}
    c_1y_1 + c_2y_2 &= b_0 \\
    c_1y_1' + c_2y_2' &= b_1
\end{align*}

It can be written as:
\[
    \begin{bmatrix}
       y_1 & y_2 \\
       y_1' & y_2'  \\
    \end{bmatrix}
    \begin{bmatrix}
         c_1 \\
         c_2 \\
    \end{bmatrix}
    =
    \begin{bmatrix}
         b_0 \\
         b_1 \\
    \end{bmatrix}
\]

To make it solvable, we need to have the determinant of the first matrix 0. 

This is called the \textbf{Wronskian} of \(f\) and \(g\):
\begin{definition}[Wronskian]
    \(W = \begin{vmatrix}
        f &  g \\
        f' & g'  \\
    \end{vmatrix}
    = fg' - f'g
    \) 
\end{definition}  

\begin{theorem}[Wronskian of Solutions]
    Suppose that \(y_1\) and \(y_2\) are two solutions of the homogeneous second-order linear equation:
    \[
        y'' + p(x)y' + q(x)y = 0
    \]  
    on an open interval \(I\) on which \(p\) and \(q\) are continuous.   
    \begin{enumerate}
        \item If \(y_1\) and \(y_2\) are linearly dependent, then \(W(y_1, y_2) \equiv 0\) on \(I\).    
        \item If \(y_1\) and \(y_2\) are linearly independent, then \(W(y_1, y_2) \neq 0\) at each point of \(I\).    
    \end{enumerate}
\end{theorem}

\begin{theorem}[General Solutions of Homogeneous Equations]
    Let \(y_1\) and \(y_2\) be 2 linearly independent solutions of the homogeneous equation
    \[
        y'' + p(x) y' + q(x) y = 0
    \]  
    with \(p\) and \(q\) continuous on the open interval \(I\).   
    If \(Y\) is any solution on \(I\), then there exist numbers \(c_1\) and \(c_2\) such that
    \[
        Y(x) = c_1y_1(x) + c_2y_2(x)
    \]    
    for all \(x\) in \(I\).  
\end{theorem}

\subsection{Linear Second-Order Equations with Constant Coefficients}

For the general homogeneous equation:
\[
    ay'' + by' + c = 0
\]
Also we observe that :
\[
    (e^{rx})' = re^{rx}
\]
and
\[
    (e^{rx})'' = r^2 e^{rx}
\]

Any derivative of \(e^{rx}\) is a constant multiple of \(e^{rx}\), we can substitute that in the general form equation:
\[
    ar^2e^{rx} + bre^{rx} + ce^{rx} = 0
\]  

We can get:
\[
    ar^2 + br + c = 0
\]
This quadratic equation is called the \textbf{characteristic equation}. 
We care about whether the characteristic equation has 2 distinct roots.

\begin{theorem}[Distict Real Roots]
   If the roots \(r_1\) and  \(r_2\) are real and distinct, then:
   \[
    y(x) = c_1 e^{r_1x} + c_2 e^{r_2x}
   \] 
   is a general solution of the equation.
\end{theorem}

\begin{example}
   Find the general solution of
   \[
    2 y'' - 7y' + 3y = 0
   \] 
\end{example}
\begin{proof}[solution]
   The characteristic equation is:
   \[
        2r^2 - 7r + 3 = 0
   \] 
   We got 2 solutions \(r_1 = 1/2\) and \(r_2 = 3\) are real and distinct, so the general solution is:
   \[
    y(x) = c_1 e^{x/2} + c_2 e^{3x}
   \]  
\end{proof}

\begin{example}
    The DE \(y'' + 2y' = 0\) has characteristic equation:
    \[
        r^2 + 2r = r(r+2) = 0
    \] 
    with 2 distinct real roots \(r_1 = 0\) and \(r_2 = -2\), 
    then we have the general solution:
    \[
        y(x) = c_1 + c_2 e^{-2x}
    \]
\end{example}

\begin{theorem}[Repeated Roots]
   If the characteristic equation has equal (necessarily real) roots \(r_1 = r_2\), then
   \[
        y(x) = (c_1 + c_2x)e^{r_1x} 
   \]  
   is the general solution.
\end{theorem}

\begin{example}
    To solve initial value problem:
    \[
        y'' + 2y' + y = 0
    \]
    and \(y(0) = 5\) and \(y'(0) = -3\)  
\end{example}
\begin{proof}[solution]
    The characteristic equation is: 
    \[
        r^2 + 2r + 1 = 0
    \]
    We have 2 same roots \(r_1 = r_2 = 1\) 
    According to the above theorem, we have:
    \begin{align*}
        &y(0) = c_1 = 5 \\
        &y'(0) = -c_1 + c_2 = -3
    \end{align*}
    which implies that \(c_1 = 5\) and \(c_2 = 2\).   
    Thus the desired particular solution of the initial value problem is:
    \[
        y(x) = 5e^{-x} + 2xe^{-x}
    \]
\end{proof}

\section{General Solutions of Linear Equations}

Second-order linear equations generalizes in a natural way to the general \(n\)\textbf{th-order linear} differential equation of the form:
\[
    P_0(x)y^{(n)} + 
    P_1(x)y^{(n-1)} + 
    \cdots +
    P_{n-1}(x) y' + 
    P_n(x) y = F(x) \tag{1}
\]  

There are some continuity and not equal to 0 assumptions regarding (1).
By divide both sides by \(P_0(x)\), we have:
\[
    y^{(n)} + 
    p_1(x)y^{(n-1)} + 
    \cdots +
    p_{n-1}(x) y' + 
    p_n(x) y = f(x) \tag{2}
\]
and we can get its associated homogeneous form:
\[
    y^{(n)} + 
    p_1(x)y^{(n-1)} + 
    \cdots +
    p_{n-1}(x) y' + 
    p_n(x) y = 0 \tag{3}
\]

\begin{theorem}[Principle of Superposition for Homogeneous Equations]
   Let \(y_1, y_2, \cdots, y_n\)  be \(n\)  solutions of (3).  
   The linear combination is also the solution for (3):
\[
    y = c_1y_1 + c_2y_2 + \cdots + c_ny_n
\]
\end{theorem}

\begin{theorem}[Existence and Uniqueness for Linear Equations]
    Suppose the functions \(p_1, p_2 \cdots, p_n\) and \(f\) are continuous on the open interval \(I\)  containing the point \(a\).   
    Then given \(n\) numbers \(b_0, b_1, \cdots, b_n-1\), the nth-order linear equation (2) has a unique (that is, one and only one) solution on the entire interval \(I\) that satisfies the \(n\) initial conditions:
    \[
        y(a) = b_0, y'(a) = b_1, \cdots, y^{(n-1)}(a) = b_{n-1}
    \]  
\end{theorem}
