\documentclass[a4paper]{report}
\usepackage{graphicx}
\graphicspath{ {./Figures/} }

\AtBeginDocument{\RenewCommandCopy\qty\SI}
\input{~/Documents/workspace/Latex-Template/Note/header.tex}
\usepackage[stable]{footmisc}
\title{Theory Of Computation}
\author{Muyao Xiao}
\date{Nov. 2024}


\thispagestyle{empty}
\addbibresource{reference.bib}

\begin{document}

\maketitle

\begin{abstract}
    The note is taken by studying 18.404J/6.5400 \textit{Theory of Computation} course by professor Michael Sipser of MIT. The course material can be downloaded in \href{https://ocw.mit.edu/courses/18-404j-theory-of-computation-fall-2020/}{MIT OpenCourseWare}. Meanwhile, most contents in this note will also be derived from his book \textit{Introduction to the Theory of Computation, third edition}.

    The course is divided into 2 parts, computational theory and complexity theory. Computational theory is developed during 1930s - 1950s. It concerns about what is computable. This note will be focused on the first part.

\begin{eg}
    Program verification, mathematical truth
\end{eg}

\begin{eg}[Models of Computation]
    Finite automata, Turing machines, \(\cdots\) 
\end{eg}
\end{abstract}

\tableofcontents

\lec{1}{3}

\pagestyle{plain}
\printbibliography{}

\end{document}