\chapter{Decision Problems for Automata and Grammars}

We demonstrate certain problems can be solved algorithmically and others that cannot. Our objective is to explore the limits of algorithm solvability.

The importance of studying unsolvability:
\begin{enumerate}
    \item Knowing a problem is algorithmically unsolvable \textit{is} useful because you realize that the problem must be simplified or altered before you can find an algorithm solution.
    \item Even you know the problem is solvable, a glimpse of the unsolvable can stimulate your imagination and help you gain an important perspective on computation.
\end{enumerate}

\begin{remark}[Computational Problem]
    In theoretical computer science, a computational problem is one that asks for a solution in terms of an algorithm.(\href{https://en.wikipedia.org/wiki/Computational_problem}{Wikipedia})
\end{remark}

\begin{example}
    Some types of computational problem: decision problems, search problems, counting problems, optimization problems.

    Questions like "what is the meaning of life?" and "do I look good in this outfit" are not computational problems.

    (\href{https://nus-cs1010.github.io/1819-s1/02-algo.html}{Unit 2: Computational Problems and Algorithms})
\end{example}


\section{Acceptance Problem for DFAs}
\begin{remark}
    Sometimes distinguish a machine that is looping from one that is merely taking a long time is difficult. For this reason, we prefer Turing machines that halt on all inputs, such machines never loop.

    Those machines are called \textbf{deciders}. 
\end{remark}

The \textbf{acceptance problem} for DFAs of testing whether a particular DFA accepts a given string can be expressed as a language, \(A_{DFA}\). The language contains the encodings of all DFAs together with strings that the DFAs accept. 

Why we are representing a pair like \(\langle B, w \rangle\) as a language? Because this is convenient as we have already set up terminology for dealing with languages. 

\begin{theorem}
    Let \(A_{DFA}\) = \{\(\langle B, w \rangle\) | \(B\)  is a DFA and \(B\)  accepts w\}

    \(A_{DFA}\) is decidable 
\end{theorem}

This problem of "testing whether a DFA B accepts an input w" is the same with "whether a combination <B, w> is a member of language \(A_{DFA}\)".

In this way, we formulate the computational problem into "testing the membership in the language".

If the language is decidable, meaning we can use a TM to decide this problem in finite time (by the definition).

This theorem tells that the computational problem "testing whether a given finite automaton accepts a given string" is decidable.

\begin{proof}
    \(M = \)"On input \(\langle B, w \rangle\), where \(B\) is a DFA and \(w\) is a string:

    \begin{enumerate}
        \item Simulate B on input \(w\)
        \item If the simulation ends in an accept state, \(accept\), If it ends in a nonaccepting state, \(reject\).   
    \end{enumerate}
    " 
\end{proof}

\section{Acceptance Problem for NFAs}
\begin{theorem}
    Let \(A_{NFA}\) = \{\(\langle B, w \rangle\) | \(B\)  is a NFA and \(B\)  accepts w\}

\end{theorem}
\begin{proof}
    Given TM \(D_{A-NFA}\) that decides \(A_{NFA}\).  

    \(D_{A-NFA}\) = "On input \(\langle B, w \rangle\)
        \begin{enumerate}
            \item Convert NFA B to equivalent DFA B'
            \item Run TM \(D_{A-DFA}\) on input \(\langle B', w \rangle\)
            \item \(Accept\) if \(D_{A-DFA}\) accepts, \(Reject\) if not.
        \end{enumerate} 
        " 
\end{proof}

\section{Emptiness Problem for DFAs}
\begin{theorem}\label{theorem: empty}
    Let \(E_{DFA}\) = \{ \(\langle B \rangle \)| \(B\) is a DFA and \(L(B) = \emptyset\)\}.

    \(E_{DFA}\) is decidable. 
\end{theorem}

The problem is like asking us is it possible to write an algorithm to decide if B is a super dumb DFA and does not accept any string.

\begin{proof}
    Given TM \(D_{E-DFA}\) that decides \(E_{DFA}\).  

    \(D_{E-DFA}\) = "On input \(\langle B \rangle\)  
        \begin{enumerate}
            \item Mark start state
            \item Repeat until no new state is marked:
            
            Mark every state that has an incoming arrow from a previously marked state.
            \item \(Accept\) if no accept state is marked. \(Reject\) if some accept state is marked.  
        \end{enumerate}
        "
\end{proof}

\section{Equivalence Problem for DFAs}
\begin{theorem}
    Let \(EQ_{DFA}\) = \{ \(\langle A, B \rangle\) | A and B are DFAs and \(L(A) = L(B)\)\} 

    \(EQ_{DFA}\) is decidable.
\end{theorem}

This problem mean whether we have an algorithm can decide 2 DFAs can accept same set of strings. 

\begin{proof}
    Given TM \(D_{EQ-DFA}\) that decides \(EQ_{DFA}\).  

    \(D_{EQ-DFA}\) = "On input \(\langle A, B \rangle\) [IDEA: Make DFA C that accepts w where A and B disagree]
    \begin{enumerate}
        \item Construct DFA C where \(L(C)= (L(A) \cap \overline{L(B)}) \cup (\overline{L(A)} \cap L(B))\) (\(L(C)\) is symmetric difference, we try to prove it is an empty set)
        \item Run \(D_{E-DFA}\) on \(\langle C \rangle\) (we can do that because of \hyperref[theorem: empty]{the empty decider})
        \item  \(Accept\) if \(D_{E-DFA}\) accepts, \(Reject\) if not.   
    \end{enumerate}
    "
\end{proof}

\section{Acceptance Problem for CFGs}
\begin{theorem}
    Let \(A_{CFG}\) = \{\(\langle G, w \rangle \)| \(G\) is a CFG and \(w \in L(G)\)\} 

    \(A_{CFG} is decidable \) 
\end{theorem}
This is the same to ask does G generate w.
\begin{proof}
    The intuition is that we can check all the strings this CFG has generated and check if w is one of them. But the problem is that we need to know that this process has a bound.   

    To prove this we need another theorem that has been proved:
    \begin{theorem}[Chomsky Normal Form(CNF)] \label{theorem: Chomsky}
        CNF only allows rules:
        \begin{align*}
            A &\rightarrow BC\\
            B &\rightarrow b
        \end{align*}
    \end{theorem}
    \begin{lemma}
        Can convert every CFG into CNF.
    \end{lemma}
    \begin{lemma}
        If \(H\) is in CNF and \(w \in L(H)\) then every derivation of \(w\) has \(2 |w| - 1\) steps.  
    \end{lemma}

    The second lemma is important as it assures us that for any \(w\) we can using finite steps to generate it if it can be generated using the CFG.

   Give TM \(D_{A-CFG}\) that decides \(A_{CFG}\).  

   \(D_{A-CFG}\) = "On input \(\langle G, w \rangle\) 
   \begin{enumerate}
    \item Convert G into CNF
    \item Try all derivations of length \(2|w| - 1\) 
    \item Accept if any generate w, Reject if not
   \end{enumerate}
\end{proof}

\begin{corollary}
    Every CFL is decidable.
\end{corollary}

\section{Emptiness Problem for CFGs}

\begin{theorem}
    Let \(E_{CFG}\) = \{ \(\langle G \rangle \) | G is a CFG and \(L(G) = \emptyset\)\} 

    \(E_{CFG}\) is decidable. 
\end{theorem}
\begin{proof}
    \(D_{E-CFG}\) = "On input \(\langle G \rangle \) [IDEA: work backwards from terminals]:
    \begin{enumerate}
        \item \textcolor{blue}{Mark} all occurrences of terminals in G
        \item Repeat until no new variables are marked

        Mark all occurrences of variable A if\\
        \(A \rightarrow B_1B_2\cdots B_k\) is a rule and all \(B_i\) were already marked. 
        \item Reject if the start variable is marked. Accept if not
    \end{enumerate} 
\end{proof}

\begin{example}[How the mark process work]
    Suppose we have: 
    \begin{align*}
        S &\rightarrow RTa\\
        R &\rightarrow Tb\\
        T &\rightarrow a\\
    \end{align*}

    Step1: we can only mark T as we know that T generates only terminals
    \begin{align*}
        S &\rightarrow RT\textcolor{blue}{a}\\
        R &\rightarrow T\textcolor{blue}{b}\\
        \textcolor{blue}{T} &\rightarrow \textcolor{blue}{a}\\
    \end{align*}

    Step2: because of step 1, we can also mark R, because we know T is marked.
     \begin{align*}
        S &\rightarrow R\textcolor{blue}{Ta}\\
        \textcolor{blue}{R} &\rightarrow \textcolor{blue}{Tb}\\
        \textcolor{blue}{T} &\rightarrow \textcolor{blue}{a}\\
    \end{align*}

    And so on...
\end{example}

\section{Equivalence Problem for CFGs}
\begin{theorem}
    Let \(EQ_{CFG}\) = \{ \(\langle G, H \rangle \) | G, H are CFGs and \(L(G) = L(H)\)\}  

    \(EQ_{CFG}\) is \textbf{NOT} decidable. 
\end{theorem}
We'll do the proof in next lecture.

\begin{theorem}
    Let \(AMBIG_{CFG}\) = \{ \(\langle G \rangle\)| G is an ambiguous CFG\}  

    \(AMBIG_{CFG}\) is \textbf{NOT} decidable.
\end{theorem}

Homework.

\begin{remark}
    Why we can not follow the same technique we used to show \(EQ_{DFA}\) is decidable?
    
    Because CFLs are not closed under complementation and intersection. We can't make those symmetric difference.
\end{remark}

\section{Acceptance Problem for TMs}
\begin{theorem}
    Let \(A_{TM}\) = \{ \(\langle M, w \rangle\) M is a TM and M accepts w\}  

    \(A_{TM}\) is \textbf{NOT} decidable. 
\end{theorem}

Will be proved in next lecture

\begin{theorem}
    \(A_{TM}\) is T-recognizable. 
\end{theorem}
\begin{proof}
    TM \(U\) recognizes \(A_TM\)  

    \(U\) = "On input \(\langle M, w \rangle\)  
    \begin{enumerate}
        \item Simulate M on input w
        \item Accept if M halts and accepts
        \item Reject if M halts and rejects
        \item \st{Reject if M never halts} (not a legal TM action, because we cannot tell). 
    \end{enumerate}
    "
\end{proof}
