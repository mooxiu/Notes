\chapter{NP-Completeness}

\section{SAT Cont.}
\begin{example}[\(\leq_P\) Example]
\end{example}

\begin{definition}
    A Boolean formula \(\phi\) is in \underline{Conjunctive Normal Form} (CNF) if it has the form
    \[
        \phi = (x \lor \bar{y} \lor z) \land (\bar{x} \lor \bar{s} \lor z \lor u) \land \cdots \land (\bar{z} \lor \bar{u})
    \]

    \textbf{Literal} : a variable or a negated variable

    \textbf{Clause}: an OR of literals (the thing in a brace)
    
    \textbf{CNF}: an AND of clauses(all of them).
    
    \textbf{3CNF}: a CNF with exactly 3 literals in each clause 

    \(3SAT\) = \{ \(\langle \phi \rangle | \phi\) is a staisfiable 3CNF formula\} 
\end{definition}

\begin{definition}
    A \(k-clique\) in a graph is a subset of \(k\) nodes all directly connected by edges.   

    \begin{example}
       % TODO: show graphs 
    \end{example}

    \(CLIQUE\) = \{\(\langle G, k \rangle\) | graph G contains k-clique\} 
\end{definition}

We will show \(3SAT \leq_P CLIQUE\) 

\begin{theorem}
    \(3SAT \leq_P CLIQUE\) 
\end{theorem}
\begin{proof}
    IDEA: \(\phi\) is satisfiable \(\iff\) \(G\) has a \(k-clique\)        

    Construction to make the transformation from \(3SAT\) to \(k-CLIQUE\):
    \begin{remark}[Transformation]
        % todo: show the construction and graphs
    \end{remark}

    \begin{lemma}[Forward]
        \(\phi\) is satisfiable \(\implies\) \(G\) has a \(k-clique\).   
    \end{lemma}
    \begin{proof}
        Take any satisfying assignment to \(\phi\). Pick 1 true literal in each clause. 

        The corresponding nodes in \(G\) are a \(k-clique\) because they don't have forbidden edges.  
    \end{proof}

    \begin{lemma}[Backward]
        \(G\) has a \(k-clique\) \(\implies\) \(phi\) is satisfiable.
    \end{lemma}
    \begin{proof}
        Take any \(k-clique\) in \(G\), it must have 1 node in each clause.  

        Set each corresponding literal TRUE. That gives a satisfying assignment to \(\phi\). 
    \end{proof}

    Notice that the reduction (transformation) \(f\) is computable in polynomial time. 
\end{proof}

\begin{corollary}
    \(CLIQUE \in P \implies 3SAT \in P\) 
\end{corollary}

\begin{remark}
    Does the theorem require 3 literals per clause?

    The answer is no, it works for any size clause.
\end{remark}


\section{NP-Completeness}

\begin{definition}
    Language \(B\) is  \underline{NP-Complete}  if
    \begin{enumerate}
        \item \(B \in NP\) 
        \item For all \(A \in NP\), \(A \leq_P B\)  
    \end{enumerate}
\end{definition}

If \(B\) is NP-complete and \(B \in P\) then P = NP. 

\begin{theorem}[Cook-Levin Theorem]
    \(SAT\) is NP-complete 
\end{theorem}
\begin{proof}
    Next lecture. Assume it is true for now.
\end{proof}

% TODO: show the link list of reduction

\begin{note}
    To show some language \(C\) is NP-complete, show \(3SAT \leq_P C\). 
\end{note}


\textbf{Importance of NP-complete}:
\begin{enumerate}
    \item Showing \(B\) is NP-complete is evidence of computational intractability.
    (Show a problem is NP-complete is a powerful evidence that it is not P)
    \item Gives a good candidate for proving \(P \neq NP\).  
\end{enumerate} 

\begin{theorem}
    \(HAMPATH\) is NP-complete. 
\end{theorem}
\begin{proof}
    Show \(3SAT \leq_P HAMPATH\) (assume 3SAT is NP-Complete) 

    IDEA: simulate variables and clauses with "gadgets".


\end{proof}

\begin{remark}
    Would this construction show the undirected Hamilton path problem is NP-complete?

    No, the construction depends on \(G\) being directed. 
\end{remark}

\section{Summary}
How to do polynomial reduction.