\chapter{TM Variants, the Church-Turing Thesis}

\section{Variants of the Turing Machine Model}

\subsection{Multi-tape TMs}
\begin{theorem}
    \(A\) is T-recognizable iff some multi-tape TM recognizes A.
\end{theorem}
\begin{proof}
    
\end{proof}


\subsection{Nondeterministic TMs}

A \(Nondeterministic TM\)(NTM) is similar to a Deterministic TM except for its transition function \(\delta: Q \times \Gamma \rightarrow \P(Q \times \Gamma \{ L, R \} )\).

\begin{theorem}
    \(A\) is T-recognizable iff some NTM recognizes \(A\).  
\end{theorem}

\subsection{Enumerators}

\begin{definition}
    A \underline{Turing Enumerator} is a deterministic TM with a printer.
\end{definition}

\begin{theorem}
    \(A\) is T-recognizable iff \(A = L(E)\) for some T-enumerator E. 
\end{theorem}

\section{Church-Turing Thesis}


\section{Notation for encodings and TMs}