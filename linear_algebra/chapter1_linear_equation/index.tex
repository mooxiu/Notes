\chapter{Linear Equations}

\section{Solution Sets of Linear Systems}

\subsection{Homogeneous Linear Systems}

A system of linear equations is said to be \textbf{homogeneous} if it can be written in the form \(Ax = 0\).  

\begin{theorem}
    The homogeneous equation \(Ax == 0\) has a nontrivial solution if and only if the equation has at least one \textbf{free variable}   
\end{theorem}

I am not going to prove this, but we can understand this through an example:

\[
    \begin{bmatrix}
        3 & 5 & -4 &  0 \\
        0 & 3 & 0 &  0 \\
        0 & 0 & 0 &  0 \\
    \end{bmatrix}
\]
in this case, the 3rd line we have a free \(x_3\), thus it can be any value, so we can make sure we have nontrivial solution. 

But we should notice the theorem is talking about the solution to \(Ax = 0\) not the solution to \(Ax = b\). For a nonzero \textbf{b}, it's obviously possible to have a nontrivial solution even we do not have a free variable:

\[
\begin{bmatrix}
    1 & 0 & 1   \\
    0 & 1 & 2  \\
\end{bmatrix}
\]
obviously, in the above example, we can have a solution \(\{1, 1\}\) to it. 


