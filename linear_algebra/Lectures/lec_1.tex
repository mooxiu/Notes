\chapter{Linear Equations}

\section{Solution Sets of Linear Systems}

\subsection{Homogeneous Linear Systems}

\begin{definition}
    A variable is a \textbf{basic variable} if it corresponds to a pivot column.  Otherwise, the variable is known as a \textbf{free variable}
\end{definition}

Contents about basic variable and free variable comes from \href{https://www.math.wsu.edu/faculty/hudelson/free.html}{here}.


\begin{eg}
    \begin{align*}
        x_1 + 2x_2 -x_3 &= 4\\
        2x_1 - 4x_2 &= 5
    \end{align*}
    
    We can make an augmented matrix:
    \[
        \begin{bmatrix}
            1 & 2 & -1 &  4 \\
            2 & -4 & 0 &  5 \\
        \end{bmatrix} 
    \]

    and it can be reduced to 
   \[
        \begin{bmatrix}
            1 & 2 & -1 &  4 \\
            0 & -8 & 2 &  -3 \\
        \end{bmatrix} 
    \]
    here the pivot positions, that is the first and second columns are pivot columns, so \(x_1\) and \(x_2\) are basic variables. The third column is not a pivot column, so \(x_3\) is a free variable. 

\end{eg}


A system of linear equations is said to be \textbf{homogeneous} if it can be written in the form \(Ax = 0\).  

\begin{theorem}
    The homogeneous equation \(Ax == 0\) has a nontrivial solution if and only if the equation has at least one \textbf{free variable}   
\end{theorem}

I am not going to prove this, but we can understand this through an example:

\[
    \begin{bmatrix}
        3 & 5 & -4 &  0 \\
        0 & 3 & 0 &  0 \\
        0 & 0 & 0 &  0 \\
    \end{bmatrix}
\]
in this case, the 3rd line we have a free \(x_3\), thus it can be any value, so we can make sure we have nontrivial solution. 

But we should notice the theorem is talking about the solution to \(Ax = 0\) not the solution to \(Ax = b\). For a nonzero \textbf{b}, it's obviously possible to have a nontrivial solution even we do not have a free variable:

\[
\begin{bmatrix}
    1 & 0 & 1   \\
    0 & 1 & 2  \\
\end{bmatrix}
\]
obviously, in the above example, we can have a solution \(\{1, 1\}\) to it. 


