\chapter{Summary of Decompositions}

\section{LU Factorization} \label{sec: LU Factorization}

\begin{definition}
    \[
        A = 
        \begin{bmatrix}
            1 & 0 & 0 &  0 \\
            * & 1 & 0 &  0 \\
            * & * & 1 &  0 \\
            * & * & * &  1 \\
        \end{bmatrix}
        \begin{bmatrix}
            \blacksquare & * & * & * & *  \\
            0 & \blacksquare  & * & * & *  \\
            0 & 0 & 0 & \blacksquare & *  \\
            0 & 0 & 0 & 0 & 0 \\
        \end{bmatrix}
    \]

    where:
    \begin{itemize}
        \item \(A\) is an \(m \times n\)matrix that can be row reduced to echelon form, without row interchanges 
        \item \(L\) is an \(m \times m\) lower triangular matrix with \(1\)'s on the diagonal
        \item \(U\) is an \(m \times n\) echelon form of \(A\)     
    \end{itemize}
\end{definition}

\begin{remark}[Why it is useful]
   It makes solving \(Ax = b\) less computation:
   
   Suppose we already have the factorized \(A = LU\), the transformation from \(x\) to \(b\) can be regarded as from:    
   \[
    x \xrightarrow{A} b
   \]
   to 
   \[
   x \xrightarrow{U}  y \xrightarrow{L} b
   \]
    Notice the ordering of \(U\) and \(L\) because \(Ax = (LU)x = L(Ux)\).   

    Intuitively, consider the shape of the \(L\) and \(L\), they make the reduce of the augmented matrix easier.  

    Therefore, even we have to reduce twice, it can still be of less operations.
\end{remark}

\begin{note}[How to solve linear system]
    If \(A = LU\), trying to solve \(Ax = b\). 
    Note that \(Ux = y\) and \(Ly = b\).

    \begin{enumerate}
        \item Row reduce \([L \quad b]\) we have \([I \quad y]\)  
        \item Row reduce \([U \quad y]\) we have \([I \quad x]\)  
    \end{enumerate}
\end{note}

\begin{problem}[Book Chapter 2.5 Exercise 19]
    Let \(A\) be a lower triangle \(n \times n\) matrix with nonzero entries on the diagonal. 
    Show that \(A\) is invertible and \(A^{-1}\) is lower triangle.  
\end{problem}
\begin{proof}[Solution]
    Show \(A\) is invertible:  
    Show that \([A \quad I]\) can be reduced to \([I \quad A^{-1}]\)... this is easy.

    Show \(A^{-1}\) is lower triangle: in the previous process, we only need to multiply the above row and add to the below row.
\end{proof}


\section{QR Factorization}
\begin{problem}[Book Chapter 2.5 Exercise 24]
    Suppose \(A = QR\), where \(Q\) and \(R\) are \(n \times n\), 
    \(R\) is invertible and upper triangle, 
    and \(Q\) has the property that \(Q^TQ = I\).   

    Show that for each \(b\) in \(R^n\), the equation \(Ax = b\) has a unique solution.   
    What computations with \(Q\) and \(R\) will produce the solution.  
\end{problem}
\begin{proof}[Solution]
    \begin{align*}
        Ax &= b \\
        QR x &= b \\
        Q^T Q R x &= Q^T b \\
        R x &= Q^T b \\
        R^{-1} R x &= R^{-1} Q^T b
    \end{align*}

    \(R\) is a triangle and we already it is invertible; \(Q\) we don't even need to do the inverse, just need to transpose it.  
\end{proof}


\section{SVD Decomposition}
\begin{problem}[Book Chapter 2.5 Exercise 25]
    Suppose \(A = UDV^T\), where \(U\) and \(V\) are \(n \times n\) matrices with the property that \(U^T U = I\) and \(V^T V = I\),
    and where \(D\) is a diagonal matrix with positive numbers \(\sigma_1, \cdots \sigma_n \) on the diagonal.        

    Show that \(A\) is invertible, and find a formula for \(A ^{-1}\).  
\end{problem}
\begin{proof}[Solution]
    Reducing \([D \quad I]\) we can easily find \([I \quad D^{-1}]\), and \(D^{-1}\) is diagonal matrix with positive numbers \(1/\sigma_1, \cdots, 1/\sigma_n\).  

    \begin{align*}
        A &= UDV^T \\
        U^TA &= DV^T \\
        D^{-1} U^T A &= V^T \\
        V D^{-1} U^T A &= I
    \end{align*}

    So \(A^{-1} = VD^{-1} U^T\).
\end{proof}

\section{Spectral Factorization}
\begin{problem}[Book Chapter 2.5 Excercise 26]
    Supppose a \(3 \times 3\) matrix \(A\) admits a factorization as \(A = PDP^{-1}\),
    where \(P\) is some invertible \(3 \times 3\) matrix and \(D\) is the diagonal matrix     
    \[
        D = \begin{bmatrix}
            1 & 0 &  0 \\
            0 & 1/2 &  0 \\
            0 & 0 &  1/3 \\
        \end{bmatrix}   
    \]
    Show that this factorization is useful when computing high powers of \(A\). 
    Find fairly simple formulas for \(A^2\), \(A^3\), and \(A^k\) (\(k\) is a positive integer), using \(P\) and the entries in \(D\).   
\end{problem}
\begin{proof}[Solution]
    Easily find \(A^k = (PDP^{-1})^k = P D^k P^{-1}\).  

    Also we find power of a diagonal matrix is easy to compute:
    \(D^k = \begin{bmatrix}
        1^k & 0 & 0  \\
        0 & (1/2)^k & 0  \\
        0 & 0 & (1/3)^3  \\
    \end{bmatrix}\) 
\end{proof}

\section{Cholesky Factorization}

\section{Eigenvalue Decomposition}

\section{Schur Factorization}

