\chapter{Orthogonal}

\section{Basics}

\includepdf[pages=-]{./Draft/orthorgonal.pdf}

\section{Inner Product, Length, and Orthogonality}

\subsection{Orthogonal Complements}

\begin{definition}[Orthogonal Complements]

    If a vector z is orthogonal to every vector in a subspace W of \(\R^n\), then z is said to be \textbf{orthogonal to} W. 
    
    The set of all vectors z that are orthogonal to W is called \textbf{orthogonal complement} of W and is denoted by \(W^{\perp}\) (and read as "W perpendicular" or simply "W perp").   
\end{definition}

\begin{note}
    All vectors in \(W^{T}\) are perpendicular to all vectors in \(W\).  
\end{note}

\begin{theorem}
    Let \(A\) be an \(m \times n\) matrix.   
    The orthogonal complement of the row space of \(A\) is the null space of \(A\),
    and the orthogonal complement of the column space of \(A\) is the null space of \(A^T\):
        
    \((Row A)^{\perp} = Nul A\) and \((Col A)^{\perp} = Nul A^T\)  
\end{theorem}
\begin{proof}
    Prove \((Row A)^{\perp} = Null A\)  here:

    Suppose each row of \(A\) is \(a_1, a_2, \cdots, c_m\), and for any vector v that is perpendicular to each row (that is v orthogonal to the subspace of Row A, that is v belongs to \((Row A)^{\perp}\)), we have:
    \[
        \begin{bmatrix}
             a_1v \\
             a_2v \\
             \cdots \\
             a_mv \\
        \end{bmatrix} = 0
    \] 

    In the same time we have:
    \[
         \begin{bmatrix}
             a_1v \\
             a_2v \\
             \cdots \\
             a_mv \\
        \end{bmatrix}
        = 
        \begin{bmatrix}
             a_1 \\
             a_2 \\
             \cdots \\
             a_m \\
        \end{bmatrix}v
        = Av
    \]

    Now we know that \(Av = 0\), so all v make up the null space of A, that is \((Row A)^{\perp} = Nul A\). 
    
    Another one can be proved in the same flow.
\end{proof}